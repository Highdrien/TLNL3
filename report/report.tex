\documentclass[a4paper]{article}
\usepackage[french]{babel}
\usepackage[utf8]{inputenc}
\usepackage[T1]{fontenc}

\title{Rapport projet TLNL}
\author{Leader : Adrien ZABBAN Follower : Yanis LABEYRIE}
\date{05 novembre 2023}

\begin{document}
\maketitle
Ce document décrit le format du rapport de projet TLNL. Le rapport doit faire entre six et dix pages.
Son format est assez rigide, il est décrit ci-dessous. L'idée est de vous préparer à écrire des articles scientifiques.
Tous les aspects sont importants~:
\begin{itemize}
\item vos idées doivent être décrites avec précision
\item votre argumentation doit être claire et explicite
\item votre texte ne doit pas comporter de fautes d'orthographe
\item votre texte ne doit pas comporter de code, ce dernier sera rendu séparement
  
\end{itemize}

Le rapport est à rendre pour le 5 novembre.

\section{Introduction}

Rappeler ici les objectifs du projet dans son ensemble et la piste ou les pistes que vous avez décidé de creuser. Dites de manière très générale pourquoi ces pistes vous semblent intéressantes. Considérez que vous vous adressez à une personne qui a une bonne culture scientifique mais qui n'est pas spécialiste du domaine.

\section{Modèle Initial}

Décrire ici ce qui a été fait hors des pistes à creuser et les résultats que vous avez obtenus et eventuellement des erreurs ou limitations intéressantes du symodèle de base (ces erreurs ou limitations doivent être en relation avec la ou les pistes que vous avez décidé de creuser).

\subsection{Piste à creuser 1~: titre}

\subsubsection{Description}

Décrire ici les idées générales de la piste à creuser. Décrire ensuite les hypothèses que vous cherchez à valier ou invalider.
Vous pouvez par exemple dire que vous pensez qu'introduire une modification $X$ va avoir un effet $E$. Expliquez pourquoi, indépendamment de toute expérience.
Il est important que vous soyez le plus explicite possible quant aux hypothèses.
\subsubsection{Mise en \oe uvre}

Dire ce que vous avez du changer au modèle de base pour prendre en compte vos modifications, les problèmes auxquels vous avez été confrontés et les solutions que vous avez trouvé pour y remédier.

\subsubsection{Résultats}

Décrire les résultats que vous avez obtenu. Dites si vos hypothèses ont été vérifiées.

Si c'est le cas, donnez des exemples qui le montre.

Si ce n'est pas le cas, faire une analyse des erreurs, proposez des explications et si possible mettez les en \oe uvre dans le cas d'une nouvelle expérience. Souvent, les choses ne marchent pas du premier coup.

\subsection{Piste à creuser 2~: titre}

même structure


\section{Conclusions et perspectives}

Reprendre le problème sur lequel vous vous êtes intéressé. Décrire les hypothèses sur lesquelles vous avez travaillé. Revenez sur les résultats que vous avez obtenu et mettant en avant les points qui vous semblent intéressants.

Dites à l'issue de ce travail quelle pistes pourraient être explorées pour aller plus loin.



\end{document}
